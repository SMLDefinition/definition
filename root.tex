% this is file `root', the root of the whole semantics
% \documentclass[a4,12pt,twoside]{article}
\documentclass[letter,12pt]{article}
\title{The Definition of Standard ML}
\author{Robin Milner \and Mads Tofte \and Robert Harper \\
Laboratory for Foundations of Computer Science\\
Department of Computer Science\\
University of Edinburgh}
%Edinburgh EH9 3JZ, Scotland}
\date{%\today}
}
\newcommand{\ml}[1]{{\tt #1}}
\renewcommand{\cdots}{\mbox{$\cdot\!\cdot\!\cdot$}}

        % Core Language
\newcommand{\ttlbrace}{\mbox{\tt\char'173}}
\newcommand{\ttrbrace}{\mbox{\tt\char'175}}
\newcommand{\lttbrace}{\mbox{\tt\char'173}}
\newcommand{\rttbrace}{\mbox{\tt\char'175}}
\newcommand{\ttprime}{\mbox{\tt\char'047}}
\newcommand{\ABSTYPE}{{\tt abstype}}
\newcommand{\AND }{{\tt and}}
\newcommand{\ANDALSO }{{\tt andalso}}
\newcommand{\AS}{{\tt as}}
\newcommand{\CASE}{{\tt case}}
\newcommand{\DO}{{\tt do}}
\newcommand{\DATATYPE}{{\tt datatype}}
\newcommand{\ELSE}{{\tt else}}
\newcommand{\END}{{\tt end}}
\newcommand{\EXCEPTION}{{\tt exception}}
\newcommand{\FUN}{{\tt fun}}
\newcommand{\FN}{{\tt fn}}
\newcommand{\HANDLE}{{\tt handle}}
\newcommand{\IF}{{\tt if}}
\newcommand{\IN}{{\tt in}}
\newcommand{\INFIX}{{\tt infix}}
\newcommand{\INFIXR}{{\tt infixr}}
\newcommand{\LET}{{\tt let}}
\newcommand{\LOCAL}{{\tt local}}
\newcommand{\NONFIX}{{\tt nonfix}}
\newcommand{\OF}{{\tt of}}
\newcommand{\OP}{{\tt op}}
\newcommand{\ORELSE}{{\tt orelse}}
\newcommand{\RAISE}{{\tt raise}}
\newcommand{\REC}{{\tt rec}}
\newcommand{\THEN}{{\tt then}}
\newcommand{\TYPE}{{\tt type}}
\newcommand{\VAL}{{\tt val}}
\newcommand{\WITH}{{\tt with}}
\newcommand{\WITHTYPE}{{\tt withtype}}
\newcommand{\WHILE}{{\tt while}}

        % Modules
\newcommand{\EQTYPE}{\ml{eqtype}}
\newcommand{\FUNCTOR}{\ml{functor}}
\newcommand{\INCLUDE}{\ml{include}}
\newcommand{\SIG}{\ml{sig}}
\newcommand{\OPEN}{\ml{open}}
\newcommand{\SHARING}{\ml{sharing}}
\newcommand{\SIGNATURE}{\ml{signature}}
\newcommand{\STRUCT}{\ml{struct}}
\newcommand{\STRUCTURE}{\ml{structure}}
          % Identifier class names
\newcommand{\Var}{{\rm Var}}
\newcommand{\Con}{{\rm Con}}
\newcommand{\SCon}{{\rm SCon}}
\newcommand{\Exn}{{\rm ExCon}}
\newcommand{\TyVar}{{\rm TyVar}}
\newcommand{\ImpTyVar}{{\rm ImpTyVar}}
\newcommand{\AppTyVar}{{\rm AppTyVar}}
\newcommand{\TyCon}{{\rm TyCon}}
\newcommand{\Lab}{{\rm Lab}}
\newcommand{\StrId}{{\rm StrId}}
\newcommand{\FunId}{{\rm FunId}}
\newcommand{\SigId}{{\rm SigId}}

          % Identifier class variables
\newcommand{\id}{{\it id}}
\newcommand{\var}{{\it var}}
\newcommand{\con}{{\it con}}
\newcommand{\scon}{{\it scon}}
\newcommand{\exn}{{\it excon}}
\newcommand{\tyvar}{{\it tyvar}}
\newcommand{\atyvar}{\mbox{\tt \ttprime a}}
\newcommand{\btyvar}{\mbox{\tt \ttprime b}}
\newcommand{\ctyvar}{\mbox{\tt \ttprime c}}
\newcommand{\aityvar}{\mbox{\tt \ttprime\_a}}
\newcommand{\aetyvar}{\mbox{\tt \ttprime\ttprime a}}
\newcommand{\tycon}{{\it tycon}}
\newcommand{\lab}{{\it lab}}
\newcommand{\strid}{{\it strid}}

\newcommand{\longvar}{{\it longvar}}
\newcommand{\longcon}{{\it longcon}}
\newcommand{\longexn}{{\it longexcon}}
\newcommand{\longtycon}{{\it longtycon}}
\newcommand{\longstrid}{{\it longstrid}}

          % General classes
\newcommand{\phrase}{{\it phrase}}

          % Core Syntax Classes
\newcommand{\apexp}{{\it appexp}}
\newcommand{\atpat}{{\it atpat}}
\newcommand{\atexp}{{\it atexp}}
\newcommand{\bind}{{\it bind}}
\newcommand{\constrs}{{\it conbind}}
\newcommand{\ConBind}{{\rm ConBind}}
\newcommand{\datbind}{{\it datbind}}
\newcommand{\dec}{{\it dec}}
\newcommand{\dir}{{\it dir}}
\newcommand{\exnbind}{{\it exbind}}
\renewcommand{\exp}{{\it exp}}
\newcommand{\fvalbind}{{\it fvalbind}}
\newcommand{\handler}{{\it handler}}
\newcommand{\hanrule}{{\it hrule}}
\newcommand{\inexp}{{\it infexp}}
\newcommand{\labexps}{{\it exprow}}
\newcommand{\labpats}{{\it patrow}}
\newcommand{\labtys}{{\it tyrow}}
\newcommand{\match}{{\it match}}
\newcommand{\pat}{{\it pat}}
\newcommand{\mrule}{{\it mrule}}
\newcommand{\ty}{{\it ty}}
\newcommand{\tyseq}{{\it tyseq}}
\newcommand{\tyvarseq}{{\it tyvarseq}}
\newcommand{\typbind}{{\it typbind}}
\newcommand{\valbind}{{\it valbind}}

 % ranged over by the variables




       % Modules syntax classes

\newcommand{\datdesc}{{\it datdesc}}
\newcommand{\condesc}{{\it condesc}}
\newcommand{\exndesc}{{\it exdesc}}
\newcommand{\funbind}{{\it funbind}}
\newcommand{\fundec}{{\it fundec}}
\newcommand{\fundesc}{{\it fundesc}}
\newcommand{\funid}{{\it funid}}
\newcommand{\funsigexp}{{\it funsigexp}}
\newcommand{\funspec}{{\it funspec}}
\newcommand{\funtyp}{{\it funtyp}}
\newcommand{\longstridk}{\strid_1.\cdots.\strid_k}
\newcommand{\longtyconk}{\strid_1.\cdots.\strid_k.\tycon}
\newcommand{\topdec}{{\it topdec}}
\newcommand{\program}{{\it program}}
\newcommand{\sigid}{{\it sigid}}
\newcommand{\shareq}{{\it shareq}}
\newcommand{\sigbind}{{\it sigbind}}
\newcommand{\sigdec}{{\it sigdec}}
\newcommand{\sigexp}{{\it sigexp}}
\newcommand{\spec}{{\it spec}}
\newcommand{\strbind}{{\it strbind}}
\newcommand{\strdec}{{\it strdec}}
\newcommand{\strexp}{{\it strexp}}
\newcommand{\strdesc}{{\it strdesc}}
\newcommand{\typdesc}{{\it typdesc}}
\newcommand{\valdesc}{{\it valdesc}}
           % Core Language Phrases

           % Expressions

\newcommand{\recexp}{\mbox{$\langle\labexps\rangle$}}
\newcommand{\longlabexps}{\mbox{\lab\ \ml{=} \exp\
                          $\langle$ \ml{,} \labexps$\rangle$}}
\newcommand{\parexp}{\mbox{\ml{(} \exp\ \ml{)}}}
\newcommand{\appexp}
           {\mbox{\exp\ \atexp}}
\newcommand{\infexp}
           {\mbox{$\exp_1\ \id\ \exp_2$}}
\newcommand{\opp}
           {\mbox{$\langle\OP\rangle$}}
\newcommand{\typedexp}
           {\mbox{\exp\ \ml{:} \ty}}
\newcommand{\handlexp}
           {\mbox{\exp\ \HANDLE\ \match}}
\newcommand{\raisexp}
           {\mbox{\RAISE\ \exp}}
\newcommand{\letexp}
           {\mbox{\LET\ \dec\ \IN\ \exp\ \END}}
\newcommand{\fnexp}
           {\mbox{\FN\ \match}}

           % Matches and Handlers
\newcommand{\longmatch}
           {\mbox{\mrule\ $\langle$ \ml{|} \match$\rangle$}}
\newcommand{\longmatcha}
           {\mbox{\mrule\ \ml{|} \match}}
\newcommand{\longmrule}
           {\mbox{\pat\ \ml{=>} \exp}}
\newcommand{\longhandler}
           {\mbox{\hanrule\ $\langle$ \ml{||} \handler$\rangle$}}
\newcommand{\longhandlera}
           {\mbox{\hanrule\ \ml{||} \handler}}
\newcommand{\longhanrule}
           {\mbox{\longexn\ \WITH\ \match}}
\newcommand{\lasthanrule}
           {\mbox{? \ml{=>} \exp}}

          % Declarations
\newcommand{\valdec}
           {\mbox{\VAL\ \valbind}}
\newcommand{\valdecS}
           {\mbox{$\VAL_{\U}$\ \valbind}}
\newcommand{\typedec}
           {\mbox{\TYPE\ \typbind}}
\newcommand{\datatypedec}
           {\mbox{\DATATYPE\ \datbind}}
\newcommand{\datatypedeca}
           {\mbox{\DATATYPE\ \datbind\ $\langle\WITHTYPE\ \typbind\rangle$}}
\newcommand{\abstypedec}
           {\mbox{\ABSTYPE\ \datbind\ \WITH\ \dec\ \END}}
\newcommand{\abstypedeca}
           {\mbox{\ABSTYPE\ \datbind\ $\langle\WITHTYPE\ \typbind\rangle$}}
\newcommand{\exceptiondec}
           {\mbox{\EXCEPTION\ \exnbind}}
\newcommand{\localdec}
           {\mbox{\LOCAL\ $\dec_1\ \IN\ \dec_2$\ \END}}
\newcommand{\emptydec}
           {\mbox{\qquad}}
\newcommand{\seqdec}
           {\mbox{$\dec_1\ \langle\ml{;}\rangle\ \dec_2$}}
\newcommand{\longinfix}
           {\mbox{$\INFIX\ \langle d\rangle\ \id_1\ \cdots\ \id_n$}}
\newcommand{\longinfixr}
           {\mbox{$\INFIXR\ \langle d\rangle\ \id_1\ \cdots\ \id_n$}}
\newcommand{\longnonfix}
           {\mbox{$\NONFIX\ \id_1\ \cdots\ \id_n$}}

          % Bindings
\newcommand{\longvalbind}
           {\mbox{\pat\ \ml{=} \exp\ $\langle\AND\ \valbind\rangle$}}
\newcommand{\recvalbind}
           {\mbox{\REC\ \valbind}}
\newcommand{\longtypbind}
           {\mbox{\tyvarseq\ \tycon\ \ml{=} \ty
            \ $\langle\AND\ \typbind\rangle$}}
\newcommand{\longdatbind}
           {\mbox{\tyvarseq\ \tycon\ \ml{=} \constrs
                  \ $\langle\AND\ \datbind\rangle$}}
\newcommand{\longconstrs}
           {\mbox{$\con\ \langle\OF\ \ty\rangle\
                   \langle$ \ml{|} \constrs$\rangle$}}
\newcommand{\longerconstrs}
           {\mbox{$\con\ \langle\OF\ \ty\rangle\
                   \langle\langle$ \ml{|} \constrs$\rangle\rangle$}}
\newcommand{\generativeexnbind}
           {\mbox{$\langle\OP\rangle\exn\ \langle\OF\ \ty\rangle\
                   \langle\AND\ \exnbind\rangle$}}
\newcommand{\eqexnbind}
           {\mbox{$\langle\OP\rangle$\exn\ \ml{=} 
                  $\langle\OP\rangle\longexn\ 
                  \langle\AND\ \exnbind\rangle$}}
\newcommand{\longexnbinda}
           {\mbox{\exn\ $\langle\OF\ \ty\rangle$\ 
                  $\langle\langle\AND\ \exnbind\rangle\rangle$}}
\newcommand{\longexnbindaa}
           {\mbox{\exn\
                  $\langle\AND\ \exnbind\rangle$}}
\newcommand{\longexnbindb}
           {\mbox{\exn\ \ml{=}\ \longexn\ $\langle\AND\ \exnbind\rangle$}}
% from version 1:
%\newcommand{\longexnbind}
%           {\mbox{\exn\ $\langle$\ml{:} $\ty\rangle
%                  \langle$\ml{=} $\longexn\rangle
%                  \ \langle\AND\ \exnbind\rangle$}}
%\newcommand{\longexnbinda}
%           {\mbox{\exn\ $\langle$\ml{:} $\ty\rangle
%                  \ \langle\langle\AND\ \exnbind\rangle\rangle$}}
%\newcommand{\longexnbindaa}
%           {\mbox{\exn\
%                  $\langle\AND\ \exnbind\rangle$}}
%\newcommand{\longexnbindb}
%           {\mbox{\exn\ $\langle$\ml{:} $\ty\rangle
%                  \ $\ml{=} $\longexn
%                  \ \langle\langle\AND\ \exnbind\rangle\rangle$}}
%\newcommand{\longexnbindbb}
%           {\mbox{\exn\ \ml{=} \longexn\
%                  $\langle\AND\ \exnbind\rangle$}}

          % Patterns
\newcommand{\wildpat}{\mbox{\ml{\_}}}
\newcommand{\recpat}{\mbox{$\langle\labpats\rangle$}}
\newcommand{\wildrec}{\mbox{\ml{...}}}
\newcommand{\longlabpats}{\mbox{\lab\ \ml{=} \pat\
                          $\langle$ \ml{,} \labpats$\rangle$}}
\newcommand{\parpat}{\mbox{\ml{(} \pat\ \ml{)}}}
\newcommand{\conpat}
           {\mbox{\longcon\ \atpat}}
\newcommand{\exconpat}
           {\mbox{\longexn\ \atpat}}
\newcommand{\infpat}
           {\mbox{$\pat_1\ \con\ \pat_2$}}
\newcommand{\infexpat}
           {\mbox{$\pat_1\ \exn\ \pat_2$}}
\newcommand{\typedpat}
           {\mbox{\pat\ \ml{:} \ty}}
\newcommand{\layeredpat}
           {\mbox{\var$\langle$\ml{:} \ty$\rangle$ \AS\ \pat}}
\newcommand{\layeredpata}
           {\mbox{\var\ \AS\ \pat}}

          % Types
\newcommand{\rectype}{\mbox{$\langle\labtys\rangle$}}
\newcommand{\longlabtys}{\mbox{\lab\ \ml{:} \ty\
                          $\langle$ \ml{,} \labtys$\rangle$}}
\newcommand{\constype}
           {\mbox{\tyseq\ \longtycon}}
\newcommand{\funtype}
           {\mbox{\ty\ \ml{->} \ty$'$}}
\newcommand{\partype}{\mbox{\ml{(} \ty\ \ml{)}}}
\newcommand{\longtyseq}{\mbox{\ml{(} $\ty_1,\cdots\,\ty_k$ \ml{)}}}
\newcommand{\longtyvarseq}{\mbox{\ml{(} $\tyvar_1,\cdots,\tyvar_k$ \ml{)}}}

        % Modules Phrases

\newcommand{\emptyphrase}{\qquad}


    	% structure-level declarations

\newcommand{\singstrdec}{\mbox{$\STRUCTURE\ \strbind $}}
\newcommand{\localstrdec}{\mbox{$\LOCAL\ \strdec_1\ \IN\ \strdec_2\ \END $}}
\newcommand{\openstrdec}{\mbox{$\OPEN\ \longstrid_1\ \cdots\ \longstrid_n $}}
\newcommand{\emptystrdec}{\emptyphrase}
\newcommand{\seqstrdec}{\mbox{$\strdec_1\ \langle$\ml{;}$\rangle\ \strdec_2 $}}


        % structure bindings

\newcommand{\strbinder}
           {\mbox{$\strid\ \langle$\ml{:}$\ \sigexp\rangle$
            \ml{=} $\strexp\ \langle\langle\AND\ \strbind\rangle\rangle$}}
\newcommand{\strbindera}
           {\mbox{$\strid\ \langle$\ml{:}$\ \sigexp\rangle$
            \ml{=} $\strexp\ \langle\AND\ \strbind\rangle$}}


	% structure expressions

\newcommand{\encstrexp}{\mbox{\STRUCT\ \strdec\ \END}}
\newcommand{\funappdec}{\mbox{\funid\ \ml{(}\ \strdec\ \ml{)} }}
\newcommand{\funappstr}{\mbox{\funid\ \ml{(}\ \strexp\ \ml{)} }}
\newcommand{\letstrexp}{\mbox{\LET\ \strdec\ \IN\ \strexp\ \END}}

        % specifications

\newcommand{\valspec}{\mbox{\VAL\ \valdesc}}
\newcommand{\typespec}{\mbox{\TYPE\ \typdesc}}
\newcommand{\eqtypespec}{\mbox{\EQTYPE\ \typdesc}}
\newcommand{\datatypespec}{\mbox{\DATATYPE\ \datdesc}}
\newcommand{\exceptionspec}{\mbox{\EXCEPTION\ \exndesc}}
\newcommand{\structurespec}{\mbox{\STRUCTURE\ \strdesc}}
\newcommand{\sharingspec}{\mbox{\SHARING\ \shareq}}
\newcommand{\localspec}{\mbox{$\LOCAL\ \spec_1\ \IN\ \spec_2\ \END$}}
\newcommand{\openspec}{\mbox{$\OPEN\ \longstrid_1\ \cdots\ \longstrid_n $}}
\newcommand{\emptyspec}{\emptyphrase}
\newcommand{\seqspec}{\mbox{$\spec_1\ \langle$\ml{;}$\rangle\ \spec_2$}}
\newcommand{\inclspec}{\mbox{$\INCLUDE\ \sigid_1\ \cdots\ \sigid_n $}}


        % descriptions

\newcommand{\valdescription}
           {\mbox{\var\ \ml{:} $\ty\ \langle\AND\ \valdesc\rangle$}}

\newcommand{\typdescription}
           {\mbox{\tyvarseq\ \tycon\ $\langle\AND\ \typdesc\rangle$}}

\newcommand{\datdescription}
           {\mbox{\tyvarseq\ \tycon\ \ml{=} \condesc
             \ $\langle\AND\ \datdesc\rangle$}}

\newcommand{\condescription}
           {\mbox{$\con\ \langle\OF\ \ty\rangle\
                   \langle$ \ml{|} \condesc$\rangle$}}

\newcommand{\longcondescription}
           {\mbox{$\con\ \langle\OF\ \ty\rangle\
                   \langle\langle$ \ml{|} \condesc$\rangle\rangle$}}

\newcommand{\exndescription}
           {\mbox{\exn\ $\langle\OF\ \ty\rangle$
            \ $\langle\AND\ \exndesc\rangle$}}

\newcommand{\exndescriptiona}
           {\mbox{\exn\ $\langle\OF\ \ty\rangle$
            \ $\langle\langle\AND\ \exndesc\rangle\rangle$}}

\newcommand{\strdescription}
           {\mbox{\strid\ \ml{:} \sigexp
            \ $\langle\AND\ \strdesc\rangle$}}

        % sharing equations

\newcommand{\strshareq}{\mbox{$\longstrid_1$ \ml{=} $\cdots$
                                             \ml{=} $\longstrid_n$ }}
\newcommand{\typshareq}{\mbox{$\TYPE\ \longtycon_1$ \ml{=} $\cdots$
                                                    \ml{=} $\longtycon_n$ }}
\newcommand{\multshareq}{\mbox{$\shareq_1\ \AND\ \shareq_2$}}


	% signature expressions

\newcommand{\encsigexp}{\mbox{\SIG\ \spec\ \END}}

        % signature declarations

\newcommand{\singsigdec}{\mbox{$\SIGNATURE\ \sigbind $}}
\newcommand{\emptysigdec}{\emptyphrase}
\newcommand{\seqsigdec}{\mbox{$\sigdec_1\ \langle$\ml{;}$\rangle\ \sigdec_2 $}}


        % signature bindings

\newcommand{\sigbinder}
           {\mbox{\sigid\ \ml{=} \sigexp
            \ $\langle\AND\ \sigbind\rangle$}}


        % functor declarations

\newcommand{\singfundec}{\mbox{$\FUNCTOR\ \funbind $}}
\newcommand{\emptyfundec}{\emptyphrase}
\newcommand{\seqfundec}{\mbox{$\fundec_1\ \langle$\ml{;}$\rangle\ \fundec_2 $}}


        % functor bindings

\newcommand{\funbinder}
           {\mbox{\funid\ \ml{(} \spec\ \ml{)}
                          $\langle$\ml{:}$\ \sigexp\rangle$\ \ml{=} $\strexp$
                          $\langle\langle\AND\ \funbind\rangle\rangle$}}
\newcommand{\funbindera}
           {\mbox{\funid\ \ml{(} \spec\ \ml{)}
                          $\langle$\ml{:}$\ \sigexp\rangle$\ \ml{=} $\strexp$}}
\newcommand{\funstrbinder}
           {\mbox{\funid\ \ml{(}\ \strid\ \ml{:}\ \sigexp\ \ml{)}
                        $\langle$\ml{:}$\ \sigexp'\rangle$\ \ml{=} $\strexp$}}
\newcommand{\optfunbind}
           {\mbox{$\langle\langle\AND\ \funbind\rangle\rangle$}}
\newcommand{\optfunbinda}
           {\mbox{$\langle\AND\ \funbind\rangle$}}
\newcommand{\funstrbindera}
           {\mbox{\funid\ \ml{(}\ \strid\ \ml{)}\ \ml{=} \strexp
            \ $\langle\AND\ \funbind\rangle$}}


        %functor signature expressions

\newcommand{\longfunsigexp}
           {\mbox{\ml{(}\ \spec\ \ml{)}
                          \ml{:}\ \sigexp}}
\newcommand{\longfunsigexpa}
           {\mbox{\ml{(}\ \strid\ \ml{:}\ \sigexp\ \ml{)}
                          \ml{:}\ \sigexp$'$}}


        % functor specifications

\newcommand{\singfunspec}{\mbox{\FUNCTOR\ \fundesc}}
\newcommand{\emptyfunspec}{\emptyphrase}
\newcommand{\seqfunspec}
           {\mbox{$\funspec_1\ \langle$\ml{;}$\rangle\ \funspec_2$}}
 

        % functor descriptions

\newcommand{\longfundesc}
           {\mbox{\funid\ \funsigexp\ $\langle\AND\ \fundesc\rangle$}}


	% programs
\newcommand{\longprog}{\mbox{\topdec\ \ml{;}\ $\langle\program\rangle$}}
\newcommand{\seqprog}
 {\mbox{$\program_1\ \langle$\ml{;}$\rangle\ \program_2$}}
% **************** END SYNTAX *************************
\newcommand{\ML}{{\rm ML}}


% 	finite sets and maps (assume math mode)
%
\newcommand{\Fin}{\mathop{\rm Fin}\nolimits}
\newcommand{\Dom}{\mathop{\rm Dom}\nolimits}
\newcommand{\Ran}{\mathop{\rm Ran}\nolimits}
\newcommand{\finfun}[2]{#1\stackrel{{\rm fin}}{\to}#2}
\newcommand{\emptymap}{\{\}}
\newcommand{\kmap}[2]{\{#1_1\mapsto#2_1,\cdots,#1_k\mapsto#2_k\}}
\newcommand{\plusmap}[2]{#1 + #2}
%
\newcommand{\TyVarSet}{{\rm TyVarSet}}
%
%
%	Names     (assume math mode)
\newcommand{\TyNames}{{\rm TyName}}
\newcommand{\TyNameSets}{{\rm TyNameSet}}
\newcommand{\StrNames}{{\rm StrName}}
\newcommand{\StrNameSets}{{\rm StrNameSet}}

\newcommand{\NameSets}{{\rm NameSet}}
\newcommand{\TyNamesk}{\TyNames^{(k)}}
\newcommand{\Addr}{{\rm Addr}}
\newcommand{\Exc}{{\rm ExName}}
\newcommand{\BasVal}{{\rm BasVal}}
\newcommand{\SVal}{{\rm SVal}}
\newcommand{\BasExc}{{\rm BasExName}}
\newcommand{\BasTyp}{{\rm BasTyp}}
\newcommand{\CONT}{{\tt !}}
\newcommand{\ASS}{{\tt :=}}
\newcommand{\FAIL}{{\rm FAIL}}
\newcommand{\Fail}{{\rm FAIL}}
\newcommand{\fail}{{\rm FAIL}}
\newcommand{\APPLY}{{\rm APPLY}}

\newcommand{\A}{a}
\newcommand{\e}{\mbox{\it en}}
\newcommand{\sv}{\mbox{\it sv}}
\newcommand{\exval}{e}
\newcommand{\excs}{\mbox{\it ens}}
\newcommand{\exns}{\mbox{\it excons}} % used in the dynamic sem. of mod.
\newcommand{\f}{f}
\newcommand{\m}{m}
\newcommand{\mem}{\mbox{\it mem}}
\newcommand{\M}{M}
\newcommand{\n}{n}
\newcommand{\N}{N}
\newcommand{\p}{p}
\renewcommand{\r}{r}
\newcommand{\res}{\mbox{\it res}}
\newcommand{\s}{s}
\renewcommand{\t}{t}
\newcommand{\T}{T}
\newcommand{\U}{U}
\newcommand{\V}{v}
\newcommand{\vars}{\mbox{\it vars}}
\newcommand{\X}{X}

          % Compound Objects (Core Language)
\newcommand{\TyEnv}{{\rm TyEnv}}
\newcommand{\TE}{\mbox{$T\!E$}}
\newcommand{\TyStr}{{\rm TyStr}}

\newcommand{\ConEnv}{{\rm ConEnv}}
\newcommand{\CE}{\mbox{$C\!E$}}

\newcommand{\VarEnv}{{\rm VarEnv}}
\newcommand{\VE}{\mbox{$V\!E$}}

\newcommand{\ExnEnv}{{\rm ExConEnv}}
\newcommand{\EE}{\mbox{$E\!E$}}

\newcommand{\IntEnv}{{\rm IntEnv}}
\newcommand{\IE}{\mbox{$I\!E$}}

\newcommand{\Env}{{\rm Env}}
\newcommand{\E}{E}
\newcommand{\longE}[1]{(\SE_{#1},\TE_{#1},\VE_{#1},\EE_{#1})}

\newcommand{\StrEnv}{{\rm StrEnv}}
\newcommand{\SE}{\mbox{$S\!E$}}

\newcommand{\Str}{{\rm Str}}
\renewcommand{\S}{S}
\newcommand{\longS}[1]{(\m_{#1},(\SE_{#1},\TE_{#1},\VE_{#1},\EE_{#1}))}

\newcommand{\Int}{{\rm Int}}
\newcommand{\I}{I}

\newcommand{\Context}{\rm Context}
\newcommand{\C}{C}


\newcommand{\Record}{{\rm Record}}
\newcommand{\ExVal}{{\rm ExVal}}
\newcommand{\Fun}{{\rm Fun}}  % used?
\newcommand{\Pack}{{\rm Pack}}
\newcommand{\Closure}{{\rm Closure}}
\newcommand{\FunctorClosure}{{\rm FunctorClosure}}
\newcommand{\Match}{{\rm Match}}
\newcommand{\State}{{\rm State}}
\newcommand{\StrExp}{{\rm StrExp}}
\newcommand{\Mem}{{\rm Mem}}
\newcommand{\ExcSet}{{\rm ExNameSet}}
\newcommand{\Val}{{\rm Val}}

\newcommand{\arity}{\mathop{\rm arity}\nolimits}
\renewcommand{\k}{\mbox{$k$}}
\newcommand{\longtauk}{(\tau_1,\cdots,\tau_k)}
\newcommand{\tauk}{\mbox{$\tau^{(k)}$}}
\newcommand{\longalphak}{(\alpha_1,\cdots,\alpha_k)}
\newcommand{\alphak}{\mbox{$\alpha^{(k)}$}}
\newcommand{\alphakt}{\mbox{$\alpha^{(k)}t$}}
%
\newcommand{\thetak}{\mbox{$\theta^{(k)}$}}
\newcommand{\tk}{\mbox{$\t^{(k)}$}}
\newcommand{\typefcn}{\theta}
\newcommand{\typefcnk}{\Lambda\alphak.\tau}
\newcommand{\Type}{{\rm Type}}
\newcommand{\ConsType}{{\rm ConsType}}
\newcommand{\RecType}{{\rm RecType}}
\newcommand{\FunType}{{\rm FunType}}
\newcommand{\constypek}{\t(\tau_1,\cdots,\tau_k)}
\newcommand{\TypeScheme}{{\rm TypeScheme}}
\newcommand{\tych}{\sigma}
\newcommand{\longtych}{\forall\alphak.\tau}
\newcommand{\Abs}{\mathop{\rm Abs}\nolimits}
\newcommand{\Inter}{\mathop{\rm Inter}\nolimits}
\newcommand{\Rec}{\mathop{\rm Rec}\nolimits}
\newcommand{\TypeFcn}{{\rm TypeFcn}}
\newcommand{\TyConFcn}{\mathop{\rm TyCon}\nolimits} % used?
\newcommand{\TyVarFcn}{\mathop{\rm tyvars}\nolimits}
\newcommand{\imptyvars}{\mathop{\rm imptyvars}\nolimits}
\newcommand{\apptyvars}{\mathop{\rm apptyvars}\nolimits}
\newcommand{\scontype}{\mathop{\rm type}\nolimits}
\newcommand{\sconval}{\mathop{\rm val}\nolimits}
%
%         Compound Objects (Modules)
%
%
\newcommand{\Sig}{{\rm Sig}}
\newcommand{\sig}{\Sigma}
\newcommand{\longsig}[1]{(\N_{#1})\S_{#1}}
\newcommand{\FunSig}{{\rm FunSig}}
\newcommand{\funsig}{\Phi}
\newcommand{\longfunsig}[1]{(\N_{#1})(\S_{#1},(\N_{#1}')\S_{#1}
')}
\newcommand{\FunEnv}{{\rm FunEnv}}
\newcommand{\F}{F}

\newcommand{\SigEnv}{{\rm SigEnv}}
\newcommand{\G}{G}

\newcommand{\Basis}{{\rm Basis}}
\newcommand{\B}{B}
\newcommand{\Bstat}{B_{\rm STAT}}
\newcommand{\Bdyn}{B_{\rm DYN}}

\newcommand{\IntBasis}{{\rm IntBasis}}
\newcommand{\IB}{\mbox{$I\!B$}}


%         Selection of Components
%
\newcommand{\of}[2]{#1 \mathbin{\rm of} #2}
\newcommand{\In}{\mbox{\rm in}}
%
%         Names in Structures
\newcommand{\StrNamesFcn}{\mathop{\rm strnames}\nolimits}
\newcommand{\TyNamesFcn}{\mathop{\rm tynames}\nolimits}
\newcommand{\TyVarsFcn}{\mathop{\rm tyvars}\nolimits}
\newcommand{\NamesFcn}{\mathop{\rm names}\nolimits}
%
%         Type Schemes (assume math mode)
%
\newcommand{\cl}[2]{{\rm Clos}_{#1}#2}
%
%         Realisations (assume math mode)
%
\newcommand{\tyrea}{\varphi_{\rm Ty}}
\newcommand{\strrea}{\varphi_{\rm Str}}
\newcommand{\rea}{\varphi}
\newcommand{\longrea}{(\tyrea,\strrea)}
%
%             Support (assume math mode)
%
\newcommand{\Supp}{\mathop{\rm Supp}\nolimits}
\newcommand{\Yield}{\mathop{\rm Yield}\nolimits}
%
%           Instantiation (assume math mode)
%
\newcommand{\siginst}[3]{#1 {\geq_{#2}} #3}
\newcommand{\sigord}[3]{#1 {\geq_{#2}} #3}
\newcommand{\funsiginst}[3]{#1 {\geq_{#2}} #3}
%
%            Inference Rules
%
%
\newcommand{\ts}{\vdash}
\newcommand{\tsdyn}{\vdash_{\rm DYN}}
\newcommand{\tsstat}{\vdash_{\rm STAT}}

\newcommand{\ra}{\Rightarrow}

%            Initial Static Basis

%              Type names
\newcommand{\BOOL}{\mbox{\tt bool}}
\newcommand{\INT}{\mbox{\tt int}}
\newcommand{\REAL}{\mbox{\tt real}}
\newcommand{\NUM}{\mbox{\tt num}}
\newcommand{\EXCN}{\mbox{\tt exn}}
\newcommand{\STRING}{\mbox{\tt string}}
\newcommand{\LIST}{\mbox{\tt list}}
\newcommand{\INSTREAM}{\mbox{\tt instream}}
\newcommand{\OUTSTREAM}{\mbox{\tt outstream}}

%              Constructors
\newcommand{\FALSE}{\mbox{\tt false}}
\newcommand{\TRUE}{\mbox{\tt true}}
\newcommand{\NIL}{\mbox{\tt nil}}
\newcommand{\REF}{\mbox{\tt ref}}
\newcommand{\UNIT}{\mbox{\tt unit}}

%              Basic Values BasVal
%\newcommand{\MAP}{\mbox{\tt map}}
%\newcommand{\REV}{\mbox{\tt rev}}
%\newcommand{\NOT}{\mbox{\tt not}}
%\newcommand{\NEG}{\mbox{\verb-~-}}
%\newcommand{\ABS}{\mbox{\tt abs}}
%\newcommand{\FLOOR}{\mbox{\tt floor}}
%\newcommand{\REAL}{\mbox{\tt real}}
%\newcommand{\SQRT}{\mbox{\tt sqrt}}
%\newcommand{\SIN}{\mbox{\tt sin}}
%\newcommand{\COS}{\mbox{\tt cos}}
%\newcommand{\ARCTAN}{\mbox{\tt arctan}}
%\newcommand{\EXP}{\mbox{\tt exp}}
%\newcommand{\LN}{\mbox{\tt ln}}
%\newcommand{\SIZE}{\mbox{\tt size}}
%\newcommand{\CHR}{\mbox{\tt chr}}
%\newcommand{\ORD}{\mbox{\tt ord}}
%\newcommand{\EXPLODE}{\mbox{\tt explode}}
%\newcommand{\IMPLODE}{\mbox{\tt implode}}
%\newcommand{\REALDIV}{\mbox{\tt /}}
%\newcommand{\DIV}{\mbox{\tt div}}
%\newcommand{\MOD}{\mbox{\tt mod}}
%\newcommand{\TIMES}{\mbox{\tt  *}}
%\newcommand{\PLUS}{\mbox{\tt +}}
%\newcommand{\MINUS}{\mbox{\tt -}}
%\newcommand{\APPEND}{\mbox{\verb-@-}}
%\newcommand{\EQ}{\mbox{\verb-=-}}
%\newcommand{\NEQ}{\mbox{\verb-<>-}}
%\newcommand{\LESS}{\mbox{\verb-<-}}
%\newcommand{\GREATER}{\mbox{\verb->-}}
%\newcommand{\LEQ}{\mbox{\verb-<=-}}
%\newcommand{\GEQ}{\mbox{\verb->=-}}
%\newcommand{\COMP}{\mbox{\tt o}}

%\newcommand{\STDIN}{\mbox{\tt std\_in}}
%\newcommand{\OPENIN}{\mbox{\tt open\_in}}
%\newcommand{\INPUT}{\mbox{\tt input}}
%\newcommand{\LOOKAHEAD}{\mbox{\tt lookahead}}
%\newcommand{\CLOSEIN}{\mbox{\tt close\_in}}
%\newcommand{\ENDSTREAM}{\mbox{\tt end\_of\_stream}}
%\newcommand{\STDOUT}{\mbox{\tt std\_out}}
%\newcommand{\OPENOUT}{\mbox{\tt open\_out}}
%\newcommand{\OUTPUT}{\mbox{\tt output}}
%\newcommand{\CLOSEOUT}{\mbox{\tt close\_out}}
%\newcommand{\IOFAILURE}{\mbox{\tt io\_failure}}
\newcommand{\comment}{{\it Comment:\ }}
\newcommand{\comments}{{\it Comments:\ }}
\newcommand{\rulesec}[2]{\subsection*{{\bf#1}\hfill\fbox{$#2$}}}
%
\font\msxm=msam10
\textfont10=\msxm
\mathchardef\restrict="0A16
%
% $$f\restrict_A$$                   % Example of use
%alignment
%\halign{\indent$#$&\quad$#$&\quad$#$\hfil&\quad\parbox[t]{6cm}{\strut#\strut}\cr
 % macros
\makeindex
%
%\includeonly{intro,syncor,statcor,statmod}
%for agfa: \voffset -12mm
%4 Sept. 89, for lw16:
\advance\hoffset by -8mm
\begin{document}
\pagestyle{empty}
\maketitle
\cleardoublepage
\pagestyle{plain}
\setcounter{page}{3}
\renewcommand{\thepage}{\roman{page}}
\include{preface}
\tableofcontents
\cleardoublepage
\pagestyle{headings}
\setcounter{page}{1}
\renewcommand{\thepage}{\arabic{page}}
\include{intro}
\include{syncor}
\include{synmod}
\include{statcor}
\include{statmod}
\include{dyncor}
\include{dynmod}
\include{prog}
\appendix
\include{app1}
\include{app2}
\include{app3}
\include{app4}
\include{app5}
\addcontentsline{toc}{section}{\protect\numberline{}{Index}}
\label{index-sec}
\begin{theindex}
\item \verb+()+ (0-tuple), 79, 80, 84, 86
\item \verb+(   )+, 3
\subitem in expression, 9, 10, 29, 61, 79, 83, 84
\subitem in pattern, 11, 34, 65, 80, 86
\subitem in sequence, 8, 82
\subitem in type expression, 11, 35, 86
\item \verb+[   ]+, 3, 79, 80, 84, 86
\item \verb+{   }+, 3
\subitem in atomic expression, 10, 29, 61, 84
\subitem in pattern, 11, 34, 65, 86
\subitem in record type expression, 11, 35, 86
\item \verb+(*  *)+ (comment brackets), 4, 6
\item \verb+,+ (comma), 3, 8, 79, 82, 84, 86
\item \verb+...+ (wildcard pattern row), 3, 11, 34, 36, 66, 86
\item \verb+_+ (underbar) 
\subitem wildcard pattern, 3, 34, 65, 86
\subitem in identifier, 4
\item \verb+|+, 3, 4, 84, 85
\item \verb+=+ (reserved word), 3
\item \verb+=+ (identifier and basic value), 4, 56, 90, 92
\item \verb+=>+, 3
\subitem in a match rule, 10, 84
\item \verb+->+, 3, 11, 35, 86
\item \verb+~+, 3, 4, 56, 88, 91
\item \verb+.+ (period) 
\subitem in real constants, 3
\subitem in long identifiers, 4
\item \verb+"+, 3
\item \verb+\+, 3, 4
\item \verb+!+, 4, 88, 91
\item \verb+%+, 4
\item \verb+&+, 4
\item \verb+$+, 4
\item \verb+#+, 3, 4, 79, 84
\item \verb(+(, 4, 56, 88, 90, 92
\item \verb+-+, 4, 56, 88, 90, 92
\item \verb+/+, 4, 56, 88, 90, 92
\item \verb+:+ (see also type constraint), 4
\item \verb+::+, 87--90
\item \verb+:=+ (assignment), 57, 61, 88, 90
\item \verb+<+, 4, 56, 88, 90, 92
\item \verb+>+, 4, 56, 88, 90, 92
\item \verb+<=+, 56, 88, 90, 92
\item \verb+>=+, 56, 88, 90, 92
\item \verb+<>+, 56, 88, 90, 92
\item \verb+?+, 4
\item \verb+@+, 4, 88, 90
\item \verb+'+, 4
\item \verb+^+, 3, 4, 88, 90
\item \verb+*+, 4, 5, 56, 80, 86, 88, 90, 92
\item $\emptymap$ (empty map), 22
\item $+$ (modification), 22, 60
\item $\oplus$, 24, 39
\item $\Lambda$ (in type function), 22, 24, 32
\item $\forall$ (in type scheme), 22, 25
\subitem see also generalisation 
\item $\alpha$ (see type variable) 
\item $\varrho$ (see record type) 
\item $\tau$ (see type) 
\item $\tauk$ (type vector), 23, 24
\item $\tych$ (type scheme), 23, 25, 26, 28, 33, 42, 50, 88, 89
\item $\longtych$ (see type scheme) 
\item $\rightarrow$ (function type), 23, 29, 35
\item $\downarrow$ (restriction), 69
\item $\typefcn$ (see type function) 
\item $(\theta,\CE)$ (see type structure) 
\item $\typefcnk$ (see type function) 
\item $\sig$ (see signature) 
\item $\longsig{}$ (see signature) 
\item $\funsig$ (see functor signature) 
\item $\longfunsig{}$ (see functor signature) 
\item $\tyrea$ (type realisation), 40
\item $\strrea$ (structure realisation), 41
\item $\rea$ (realisation), 41--43, 54
\item $\geq$ (see instance) 
\item $\succ$ (see generalisation and enrichment) 
\item $\ts$ (turnstile), 2, 28, 29, 45, 59, 70, 75
\item $\tsdyn$ (evaluation), 75
\item $\tsstat$ (elaboration), 75
\item $\ra$, 2, 28, 45, 59, 70, 75
\item $\langle\ \rangle$ (see options) 
\item $\langle'\rangle$, 47
\indexspace
\parbox{65mm}{\hfil{\large\bf A}\hfil}
\indexspace
\item $a$ (see address) 
\item $\Abs$ (abstype operation), 27, 31
\item {\tt abs}, 56, 88, 91
\item {\tt Abs}, 57, 91
\item $\ABSTYPE$, 3, 10, 27, 31, 80, 85
\item abstype declaration, 10, 27, 31, 85
\item addition of numbers (\ml{+}), 4, 56, 88, 90, 92
\item $\Addr$ (addresses), 55, 57
\item address ($\A$), 55
\subitem fresh, 61
\item admissibility, 40, 43
\item admit equality, 24, 27, 32, 40, 43, 49, 87, 92
\item $\AND$, 3, 17--19, 85
\item \ANDALSO, 3, 79, 84
\item appending lists (\verb+@+), 4, 88, 90
\item $\apexp$ (application expression), 82, 84
\item application, 10, 29
\subitem of basic value ($\APPLY$), 56, 62, 91
\subitem of (function) closure, 62
\subitem of value constructor, 61
\subitem of exception name, 61
\subitem of {\tt ref}, 61
\subitem of {\tt :=}, 61, 88
\subitem infixed, 10
\item application of functor (see functor application) 
\item application of type function, 24, 35
\item application expression, 82, 84
\item applicative type variable (see type variable) 
\item $\APPLY$ (see application) 
\item $\AppTyVar$ (applicative type variables), 5
\item $\apptyvars$ (free applicative type variables), 22
\item {\tt arctan}, 56, 88, 91
\item arity 
\subitem of type name, 21
\subitem of type function, 24, 50
\item arrow type (see function type expression) 
\item \AS, 3, 11, 35, 66, 86
\item assignment (\ml{:=}), 57, 61, 88, 90
\item $\atexp$ (atomic expression), 8, 10, 28, 60, 79, 84
\item atomic expression, 8, 10, 28, 60, 79, 84
\subitem as expression, 10, 29, 61
\item atomic pattern, 8, 11, 34, 65, 80, 86
\subitem as pattern, 11, 35, 66, 86
\item $\atpat$ (atomic pattern), 8, 11, 34, 65, 80, 86
\indexspace
\parbox{65mm}{\hfil{\large\bf B}\hfil}
\indexspace
\item $b$ (see basic value) 
\item $\B$ (see basis) 
\item $\B_0$ (initial basis) 
\subitem static, 87
\subitem dynamic, 90
\item bare language, 1
\item $\BasExc$ (basic exception names), 57, 90
\item basic value ($b$), 55, 56, 90--94
\item basis ($\B$), 1
\subitem static, 28, 38, 45, 75, 87
\subitem dynamic, 68, 75, 90
\subitem combined, 75
\item $\Basis$ (bases), 38, 68, 75
\item $\BasVal$ (basic values), 55, 56, 90--94
\item $\Bdyn$ (dynamic basis), 75
\item {\tt Bind} (exception), 57, 64
\item $\BOOL$, 87, 89
\item bound names, 38, 39, 41
\item $\Bstat$ (static basis), 75
\indexspace
\parbox{65mm}{\hfil{\large\bf C}\hfil}
\indexspace
\item $\C$ (context), 23--22, 28--36
\item ``{\tt Cannot open} $s$'', 93
\item \CASE, 3, 79, 84
\item $\CE$ (constructor environment), 23, 27, 33, 51
\item {\tt chr}, 56, 88, 91
\item {\tt Chr}, 57, 91
\item $\cl{}{}$ (closure of types etc.), 26, 31, 33, 49, 50
\item \verb+close_in+, 56, 89, 92, 93
\item \verb+close_out+, 56, 89, 93
\item $\Closure$ (function closures), 57
\subitem recursion, 58
\item closure rules (signatures and functors), 19, 48, 52
\item coercion of numbers (\ml{real}), 56, 88, 91
\item comments, 4, 6
\item composition of functions (\ml{o}), 88, 90
\item $\con$ (see value constructor) 
\item $\Con$ (value constructors), 4, 57
\item $\constrs$ (constructor binding), 8, 10, 33, 85
\item $\ConBind$ (constructor bindings), 8, 55
\item concatenating strings (\verb+^+), 4, 88, 90
\item $\condesc$ (constructor description), 15, 17, 18, 50, 68
\item ConDesc (constructor descriptions), 15, 68
\item $\ConEnv$ (constructor environments), 23
\item ``consing'' an element to a list (\ml{::}), 87--90
\item consistency 
\subitem of type structures, 39, 51
\subitem of semantic object, 39, 40, 51
\item constant (see also value constant and exception constant) 
\subitem special (see special constant) 
\item construction (see value construction and  exception construction) 
\item constructor binding ($\constrs$), 8, 10, 33, 85
\item constructor description, 15, 17, 18, 50, 68
\item constructor environment ($\CE$), 23, 27, 33, 51
\item $\ConsType$ (constructed types), 23
\item contents of (see dereferencing) 
\item context ($\C$), 23--22, 28--36
\item $\Context$ (contexts), 23
\item control character, 3
\item Core Language, 1
\subitem syntax, 3
\subitem static semantics, 21
\subitem dynamic semantics, 55
\item Core Language Programs, 77
\item {\tt cos}, 56, 88, 91
\item cover, 43
\item cycle-freedom, 40
\indexspace
\parbox{65mm}{\hfil{\large\bf D}\hfil}
\indexspace
\item \DATATYPE, 3, 10, 18, 31, 49, 68, 80, 85
\item datatype binding, 8, 10, 33, 85
\item datatype declaration, 10, 31, 85
\item datatype description, 15, 18, 50
\item datatype specification, 18, 49, 68
\item $\datbind$ (datatype binding), 8, 10, 33, 85
\item DatBind (datatype bindings), 8, 55
\item $\datdesc$ (datatype description), 15, 18, 50
\item DatDesc (datatype descriptions), 15, 68
\item $\dec$ (declaration), 8, 10, 31, 64, 80, 85
\item Dec (declarations), 8
\item declaration (Core), 8, 10, 31, 64, 80, 85
\subitem as structure-level declaration, 17, 46, 71
\item dereferencing (\ml{!}), 4, 88, 91
\item derived forms, 1, 7, 13, 78--81
\item {\tt Diff}, 57, 92
\item digit 
\subitem in identifier, 4
\subitem in integers and reals, 3
\item $\dir$ (fixity directive), 7, 10, 13
\item directive, 10
\item {\tt div}, 56, 88, 90, 92
\item {\tt Div}, 57, 92
\item division of reals (\ml{/}), 56, 88, 90, 92
\item \DO, 3, 79, 84
\item $\Dom$ (domain), 22
\item dynamic 
\subitem semantics (Core), 55
\subitem semantics (Modules), 68
\subitem basis (see basis) 
\indexspace
\parbox{65mm}{\hfil{\large\bf E}\hfil}
\indexspace
\item $\exval$ (exception value), 57
\item $[\exval]$ (see packet) 
\item \verb+E+\ (exponent), 3
\item $\E$ (environment) 
\subitem static, 23, 27, 28, 31, 32
\subitem dynamic, 57, 60--67, 69, 70
\item $\EE$ (see exception constructor environment) 
\item elaboration, 1, 2, 28, 45, 75
\item \ELSE, 3, 79, 84
\item empty 
\subitem declaration (Core), 10, 31, 64, 85
\subitem functor declaration, 19, 52, 73
\subitem functor specification, 19, 52
\subitem signature declaration, 17, 48, 72
\subitem specification, 18, 49, 72
\subitem structure-level declaration, 17, 46, 71
\item $\e$ (exception name), 55, 64
\item \END, 3, 10, 17, 18, 84, 85
\item \verb+end_of_stream+, 56, 89, 93
\item enrichment ($\succ$), 36, 42, 45, 47, 53
\item $\excs$ (exception name set), 57, 64
\item $\Env$ (environments), 23, 57
\item \EQTYPE, 13, 18, 49, 68
\item equality 
\subitem admit equality, 24, 27, 32, 40, 43, 49, 87, 92
\subitem maximise equality, 27, 31
\subitem on abstract types, 27
\subitem of structures (sharing), 51
\subitem of type functions (sharing), 24, 51
\subitem of type schemes, 25
\subitem of values, 24, 88, 90, 92
\subitem -principal, 43, 48, 52, 53
\subitem respect equality, 27, 32, 43
\item equality attribute 
\subitem of type name, 21, 24, 27, 40, 43, 49
\subitem of type variable, 5, 21, 24, 25
\item equality type, 24, 88
\item equality type function, 24
\item equality type specification, 18, 49, 68
\item equality type variable, 5, 21, 24
\item escape sequence, 3
\item evaluation, 1, 2, 59, 70, 75
\item $\exnbind$ (exception binding), 8, 10, 33, 64, 85
\item ExBind (exception bindings), 8
\item \EXCEPTION, 3, 10, 18, 31, 49, 64, 72, 85
\item exception binding, 8, 10, 33, 64, 85
\item exception constant ($\exn$ or $\longexn$) 
\subitem as atomic pattern, 11, 34, 65, 86
\item exception construction 
\subitem as pattern, 11, 35, 66, 86
\subitem infixed, as pattern, 7, 11, 86
\item exception constructor 
\subitem as atomic expression, 10, 28, 61, 84
\item exception constructor environment ($\EE$) 
\subitem static, 23--22, 33, 69
\subitem dynamic, 57, 64, 69
\item exception convention, 60--62, 76
\item exception declaration, 10, 31, 64, 85
\item exception description, 15, 18, 51, 73
\item exception name ($\e$), 55
\subitem fresh, 64
\item exception name set ($\excs$), 57, 64
\item exception packet (see packet) 
\item exception specification, 18, 49, 72
\item exception value ($\exval$), 57
\item $\exn$ (see exception constant or constructor) 
\item $\Exn$ (exception constructors), 4
\item $\ExnEnv$ (exception constructor environments), 23, 57
\item $\exns$ (exeption constructor set), 68, 73
\item $\exndesc$ (exception description), 15, 18, 51, 73
\item ExDesc (exception descriptions), 15
\item execution, 1, 75
\item exhaustive patterns, 36, 57
\item $\EXCN$, 29, 34, 35, 51, 87, 89
\item $\Exc$ (exception names), 55
\item $\ExcSet$ (exception name sets), 57
\item $\exp$ (expression), 8, 10, 29, 61, 79, 84
\item Exp (expressions), 8
\item {\tt exp} (exponential), 56, 88, 91
\item {\tt Exp}, 57, 91
\item expansive expression, 26
\item {\tt explode} (a string), 56, 88, 92
\item expression, 8, 10, 29, 61, 79, 84
\item expression row, 8, 10, 29, 61, 84
\item $\labexps$ (expression row), 8, 10, 29, 61, 84
\item ExpRow (expression rows), 8
\item $\ExVal$ (exception values), 57
\indexspace
\parbox{65mm}{\hfil{\large\bf F}\hfil}
\indexspace
\item $\F$ (functor environment), 38, 52, 53, 68, 73
\item $\FAIL$ (failure in pattern matching), 55, 60--63, 65--67
\item \FALSE, 87--89
\item $\finfun{}{}$ (finite map), 22
\item $\Fin$ (finite subset), 22
\item {\tt floor}, 56, 88, 91
\item {\tt Floor}, 57, 91
\item \FN, 3, 10, 11, 29, 62, 84
\item formatting character, 4
\item \FUN, 3, 78, 80, 85
\item $\funbind$ (functor binding), 15, 19, 53, 73, 78, 81
\item FunBind (functor bindings), 15
\item function ($\fnexp$), 10, 29, 62, 84
\item function declaration (see $\FUN$) 
\item function type ($\rightarrow$), 23, 29, 35
\item function type expression (\verb+->+), 11, 35, 86
\item function-value binding ($\fvalbind$), 36, 78, 80, 85
\item \FUNCTOR, 13, 19, 52, 73
\item functor application, 17, 45, 70, 81
\item functor binding, 15, 19, 53, 73, 78, 81
\item functor closure, 68, 70, 73
\item functor declaration, 15, 19, 52, 73
\subitem as top-level declaration, 19, 53, 74
\item functor description, 15, 19, 52
\item functor environment ($\F$), 38, 52, 53, 68, 73
\item functor identifier ($\funid$), 13, 17, 19
\item functor signature ($\funsig$), 38, 52--54
\item functor signature expression, 15, 19, 52, 81
\item functor signature matching, 15, 54
\item functor specification, 15, 19, 52
\item $\FunctorClosure$ (functor closures), 68
\item $\fundec$ (functor declaration), 15, 19, 52, 73
\item FunDec (functor declarations), 15
\item $\fundesc$ (functor description), 15, 19, 52
\item FunDesc (functor descriptions), 15
\item $\FunEnv$ (functor environments), 38, 68
\item $\funid$ (functor identifier), 13, 17, 19
\item $\FunId$ (functor identifiers), 13
\item $\funsigexp$ (functor signature expression), 15, 19, 52, 81
\item FunSigExp (functor signature expressions), 15
\item $\funspec$ (functor specification), 15, 19, 52
\item FunSpec (functor specifications), 15
\item $\FunType$ (function types), 23
\item $\fvalbind$ (function-value binding), 78, 80, 85
\subitem exhaustive, 36
\indexspace
\indexspace
\indexspace
\parbox{65mm}{\hfil{\large\bf G}\hfil}
\indexspace
\item $\G$ (signature environment), 38, 48, 68, 72
\item generalisation ($\succ$), 25, 28, 34--36, 42
\item generative signature expression, 17, 47, 71
\item generative structure expression, 17, 45, 70
\item grammar, 1
\subitem for the Core, 7, 82
\subitem for Modules, 14
\indexspace
\parbox{65mm}{\hfil{\large\bf H}\hfil}
\indexspace
\item \HANDLE, 3, 10, 29, 62, 84
\indexspace
\parbox{65mm}{\hfil{\large\bf I}\hfil}
\indexspace
\item $\I$ (interface), 68, 71, 72
\item $\IB$ (interface basis), 68, 69, 71--73
\item identifier ($\id$), 4, 13
\subitem alphanumeric, 4
\subitem long, 4, 77
\subitem qualified, 4
\subitem symbolic, 4
\item $\IE$ (interface environment), 68, 73
\item \IF, 3, 79, 84
\item imperative attribute, 21, 24, 25
\item imperative type, 24, 33
\item imperative type variable (see type variable) 
\item implementation, 1, 75
\item {\tt implode} (a string list), 56, 88, 92
\item $\ImpTyVar$ (imperative type variables), 5
\item $\imptyvars$ (free imperative type variables), 22, 53
\item $\In$ (injection), 22
\item \IN, 3, 10, 17, 18, 79, 84, 85
\item \INCLUDE, 13, 18, 49, 72
\item inference, 2
\item inference rules 
\subitem static semantics (Core), 28
\subitem static semantics (Modules), 45
\subitem dynamic semantics (Core), 58
\subitem dynamic semantics (Modules), 70
\item $\inexp$ (infix expression), 82, 84
\item InfExp (infix expressions), 82, 84
\item \INFIX, 3, 6, 10, 85
\item infix expression, 6, 10, 82, 84
\item infix pattern, 6, 11, 86
\item infixed identifiers, 6, 10, 13, 84--86, 88
\item \INFIXR, 3, 6, 10, 85
\item initial basis, 2, 87, 90
\item injection (\In), 22
\item {\tt input}, 56, 89, 93
\item input/output, 89, 92
\item instance ($\geq$) 
\subitem of signature, 41--43, 47, 49
\subitem of functor signature, 41, 45
\subitem in matching, 42, 43
\item $\INSTREAM$, 87, 89, 92
\item $\INT$, 87, 89
\item $\Int$ (interfaces), 68
\item $\IntBasis$ (interface bases), 68
\item integer constant, 3, 89
\item $\IntEnv$ (interface environments), 68
\item $\Inter$, 68, 71, 73
\item interaction, 1, 75
\item interface ($\I$), 68, 71, 72
\item interface basis ($\IB$), 68, 69, 71--73
\item interface environment ($\IE$), 68, 73
\item {\tt Interrupt}, 57, 76
\item {\tt Io}, 57, 93
\item irredundant patterns, 36, 57
\item {\tt it}, 80
\indexspace
\parbox{65mm}{\hfil{\large\bf L}\hfil}
\indexspace
\item L (left associative), 8, 83
\item $\lab$ (label), 4, 5
\item $\Lab$ (labels), 4, 5
\item \LET, 3
\subitem expression (Core), 10, 29, 61, 79, 84
\subitem expression (Modules), 17, 45, 70
\item letter in identifer, 4
\item lexical analysis, 6, 7
\item $\LIST$, 87, 89
\item list reversal (\ml{rev}), 88, 91
\item {\tt ln}, 56, 88, 91
\item {\tt Ln}, 57, 91
\item \LOCAL, 3
\subitem declaration (Core), 10, 31, 64, 85
\subitem declaration (Modules), 17, 46, 71
\subitem specification (Modules), 18, 49, 72
\item long identifiers (e.g. $\longexn$), 4, 77
\item {\tt lookahead}, 56, 89, 93
\indexspace
\parbox{65mm}{\hfil{\large\bf M}\hfil}
\indexspace
\item $\m$ (structure name), 21, 23--22, 38--42, 47, 51, 56
\subitem fresh, 45, 46
\item $\M$ (structure name set), 38, 45
\item \ml{map}, 88, 91
\item match ($\match$), 8, 10, 30, 63
\subitem irredundant, 36, 57
\subitem exhaustive, 36, 57
\subitem in closure, 57, 58
\item $\Match$, 8
\item {\tt Match} (exception), 57, 62
\item match rule, 8, 10, 30, 63
\item matching 
\subitem signatures (see signature matching) 
\subitem functor signatures (see functor signature matching) 
\item maximise equality, 27, 31
\item $\mem$ (memory), 57, 61, 66
\item $\Mem$ (memories), 57
\item memory ($\mem$), 57, 61, 66
\item {\tt mod}, 56, 88, 90, 92
\item {\tt Mod}, 57, 92
\item modification ($+$) 
\subitem of finite maps, 22
\subitem of environments, 22, 60
\item module, 15
\item Modules, 1
\item $\mrule$ (match rule), 8, 10, 30, 63
\item Mrule (match rules), 8
\item multiplication of numbers (\ml{*}), 56, 88, 90, 92
\indexspace
\parbox{65mm}{\hfil{\large\bf N}\hfil}
\indexspace
\item $\n$ (name, see structure name, type name and exception name) 
\item $\N$ (name set), 38, 45
\item $n$-tuple, 79, 80, 84, 86
\item name 
\subitem of structure ($\m$), 21, 23--22, 38--42, 45--47, 51, 56
\item name set ($\N$), 38, 45
\item $\NamesFcn$ (free names), 38, 39, 45, 53
\item $\NameSets$ (name sets), 38
\item Natural Semantics, 2
\item {\tt Neg}, 57, 91
\item negation of booleans (\ml{not}), 88, 91
\item negation of numbers (\verb+~+), 3, 56, 88, 91
\item \NIL, 79, 87--89
\item non-expansive expression, 26
\item \NONFIX, 3, 6, 10, 13, 85, 88
\item nonfix identifiers, 6, 10, 13, 85, 88
\item \ml{not}, 88, 91
\item \NUM, 88
\indexspace
\parbox{65mm}{\hfil{\large\bf O}\hfil}
\indexspace
\item \ml{o} (function composition), 88, 90
\item occurrence 
\subitem substructure, 38
\item $\of{}{}$ (projection), 22, 38
\item $\OF$, 3
\subitem in $\CASE$ expression, 79, 84
\subitem in constructor binding, 10
\subitem in exception binding, 10, 55
\subitem in exception description, 18, 68
\item \OP, 3, 6
\subitem on variable or constructor, 10, 11, 84--86
\subitem in constructor binding, 10, 85
\item \OPEN, 3, 10, 18, 31, 49, 64, 68, 72, 77, 81, 85
\item \verb+open_in+, 56, 89, 92, 93
\item \verb+open_out+, 56, 89, 92, 93
\item opening structures in declarations, 10, 31, 64, 85
\item opening structures in specifications, 18, 19, 49, 72
\item options, 8
\subitem first ($\langle\ \rangle$), 28, 47
\subitem second ($\langle\langle\ \rangle\rangle$), 28
\item {\tt ord} (of string), 56, 88, 92
\item {\tt Ord}, 57, 92
\item \ORELSE, 3, 79, 84
\item {\tt output}, 56, 89, 93
\item ``{\tt Output stream is closed}'' , 93
\item $\OUTSTREAM$, 87, 89, 92
\indexspace
\parbox{65mm}{\hfil{\large\bf P}\hfil}
\indexspace
\item $\p$ (see packet) 
\item $\Pack$ (packets), 57
\item packet ($\p$), 57, 60, 62, 70, 75, 76
\item parsing, 1, 75
\item $\pat$ (pattern), 8, 11, 35, 66, 80, 86
\item Pat (patterns), 8
\item $\labpats$ (pattern row), 8, 11, 34, 66, 80, 86
\item PatRow (pattern rows), 8
\item pattern, 8, 11, 35, 66, 80, 86
\subitem layered, 11, 35, 66, 86
\item pattern matching, 36, 55, 57, 66
\subitem with $\REF$, 66
\item pattern row, 8, 11, 34, 66, 80, 86
\item polymorphic 
\subitem functions, 28, 31, 34
\subitem references, 26, 31, 53, 88
\subitem exceptions, 26, 33, 51, 53
\item precedence, 8, 82
\item principal 
\subitem environment, 36, 46
\subitem equality-, 43, 48, 52, 53
\subitem signature, 43, 48, 52, 53
\item printable character, 3
\item {\tt Prod}, 57, 92
\item product type (\verb+*+), 80, 86
\item program ($\program$), 1, 75, 76
\item Program (programs), 75
\item projection ($\of{}{}$), 22, 38
\indexspace
\parbox{65mm}{\hfil{\large\bf Q}\hfil}
\indexspace
\item qualified identifier, 4
\item {\tt Quot}, 57, 92
\indexspace
\parbox{65mm}{\hfil{\large\bf R}\hfil}
\indexspace
\item $\r$ (record), 57, 61, 65, 66
\item R (right associative), 8, 83
\item \RAISE, 3, 10, 29, 30, 60, 62, 75, 84
\item $\Ran$ (range), 22
\item $\REAL$ 
\subitem the type, 87, 89
\subitem coercion, 56, 88, 91
\item real constant, 3, 89
\item realisation ($\rea$), 41--43, 54
\item $\REC$, 3, 10, 11, 32, 58, 64, 85
\item $\Rec$ (recursion operator), 58, 62, 64
\item record  
\subitem $\r$, 57, 61, 65, 66
\subitem as atomic expression, 10, 29, 61, 79, 84
\subitem as atomic pattern, 11, 34, 65, 80, 86
\subitem selector (\ml{\#}\ {\it lab}), 3, 79, 84
\subitem type expression, 11, 35, 86
\subitem type ($\varrho$), 23, 29, 34, 36
\item Record (records), 57
\item $\RecType$ (record types), 23
\item recursion (see $\REC$, $\Rec$, and $\FUN$) 
\item $\REF$ 
\subitem the type constructor, 87, 89
\subitem the type name, 24, 87--89
\subitem the value constructor, 55, 61, 66, 88, 89, 91
\item reserved words, 3, 13
\item respect equality (see equality) 
\item restrictions 
\subitem closure rules (see these) 
\subitem syntactic (Core), 11, 36
\subitem syntactic (Modules), 17
\item \ml{rev}, 88, 91
\indexspace
\parbox{65mm}{\hfil{\large\bf S}\hfil}
\indexspace
\item $\s$ (state), 57, 59, 61, 66, 70, 75, 76
\item $\S$ (structure), 23, 38, 39, 42, 45, 47, 56
\item {\SCon} (special constants), 4
\item {\scon} (see special constant) 
\item scope 
\subitem of constructor, 5, 22
\subitem of value variable, 5, 22
\subitem of fixity directive, 7, 13
\subitem of explicit type variable, 25, 31, 32
\item $\SE$ (structure environment) 
\subitem static, 23--22, 38, 42, 47, 51, 87
\subitem dynamic, 57, 69, 71, 90
\item semantic object, 2
\subitem simple (Static), 21
\subitem simple (Dynamic), 55
\subitem compound (Core, Static), 22, 23
\subitem compound (Core, Dynamic), 57
\subitem compound (Modules, Static), 38
\subitem compound (Modules, Dynamic), 68
\item sentence, 2, 28, 45, 59, 70, 75
\item separate compilation, 15, 19, 20, 54
\item sequential 
\subitem expression, 79, 84
\subitem declaration (Core), 10, 31, 64, 85
\subitem functor declarations, 19, 52, 73
\subitem functor specification, 19, 52
\subitem signature declaration, 17, 48, 72
\subitem specification, 18, 49, 72
\subitem structure-level declaration, 17, 46, 71
\item $\shareq$ (sharing equation), 15, 18, 51, 68
\item SharEq (sharing equations), 15, 68
\item sharing, 19, 20, 46, 47, 50, 51, 54
\subitem equations, 15, 18, 51, 68
\subitem specification, 18, 49
\subitem of structures, 18, 51
\subitem of types, 18, 51
\subitem multiple, 18, 51
\item \SHARING, 13, 18, 49
\item side-condition, 59, 70
\item side-effect, 70, 76
\item \SIG, 13, 17, 47, 71
\item $\Sig$ (signatures), 38
\item $\sigbind$ (signature binding), 15, 17, 48, 72
\item SigBind (signature bindings), 15
\item $\sigdec$ (signature declaration), 15, 17, 48, 72
\item SigDec (signature declarations), 15
\item $\SigEnv$ (signature environments), 38, 68
\item $\sigexp$ (signature expression), 15, 17, 47, 71
\item SigExp (signature expressions), 15
\item $\sigid$ (signature identifier), 13, 17, 47, 71
\item $\SigId$ (signature identifiers), 13
\item signature ($\sig$), 38, 39, 41--43, 47, 48, 52--54, 69
\item \SIGNATURE, 13, 17, 48, 72
\item signature binding, 15, 17, 48, 72
\item signature declaration, 15, 17, 48, 72
\subitem in top-level declaration, 19, 53, 73
\item signature environment ($\G$) 
\subitem static, 38, 48, 53
\subitem dynamic, 68, 69, 72, 74
\item signature expression, 15, 17, 47, 71
\item signature identifier, 13, 17, 47, 71
\item signature instantiation (see instance) 
\item signature matching, 42, 43, 45--47, 53
\item {\tt sin}, 56, 88, 91
\item {\tt size} (of strings), 56, 88, 91
\item $\spec$ (specification), 15, 18, 49, 72
\item Spec (specifications), 15
\item special constant (\scon), 3, 4, 22
\subitem as atomic expression, 10, 28, 60, 84
\subitem in pattern, 11, 34, 65, 86
\item special value ($\sv$), 55
\item specification, 15, 18, 49, 72
\item {\tt sqrt} (square root), 56, 88, 91
\item {\tt Sqrt}, 57, 91
\item state ($\s$), 57, 59, 61, 66, 70, 75, 76
\item $\State$, 57
\item state convention, 60, 61
\item static 
\subitem basis, 1, 28, 38, 45, 75, 87
\subitem semantics (Core), 21
\subitem semantics (Modules), 38
\item \verb+std_in+, 56, 89, 93
\item \verb+std_out+, 56, 89, 93
\item $\Str$ (structures), 23
\item $\strbind$ (structure binding), 15, 17, 47, 71
\item StrBind (structure bindings), 15
\item $\strdec$ (structure-level declaration), 15, 17, 46, 71, 77
\item StrDec (structure-level declarations), 15
\item $\strdesc$ (structure description), 15, 18, 51, 73
\item StrDesc (structure descriptions), 15
\item stream (input/output), 92
\item $\StrEnv$ (structure environments), 23, 57
\item $\strexp$ (structure expression), 15, 17, 45, 70, 81
\item $\StrExp$ (structure expressions), 15
\item $\strid$ (structure identifier), 4
\subitem as structure expression, 17, 45, 70
\item $\StrId$ (structure identifiers), 4
\item $\STRING$, 87, 89
\item string constant, 3, 89
\item $\StrNames$ (structure names), 21
\item $\StrNamesFcn$ (free structure names), 38
\item $\StrNameSets$ (structure name sets), 38
\item $\STRUCT$, 13, 17, 45, 70, 81
\item structure ($\S$ or $(\m,\E)$), 23, 38, 39, 42, 45, 47, 56
\item $\STRUCTURE$, 13, 17, 18, 46, 49, 71, 72
\item structure binding ($\strbind$), 15, 17, 47, 71
\item structure declaration, 17, 46, 71
\item structure description ($\strdesc$), 15, 18, 51, 73
\item structure environment ($\SE$) 
\subitem static, 23--22, 38, 42, 47, 51, 87
\subitem dynamic, 57, 69, 71, 90
\item structure expression ($\strexp$), 15, 17, 45, 70, 81
\item structure identifier ($\strid$), 4
\subitem as structure expression, 17, 45, 70
\item structure-level declaration ($\strdec$), 15, 17, 46, 71, 77
\subitem in top-level declaration, 19, 53, 73, 77
\item structure name ($\m$, see name) 
\item structure name set ($\M$), 38, 45
\item structure realisation ($\strrea$), 41
\item structure specification, 18, 49, 72
\item substructure, 38
\subitem proper, 38, 40
\item subtraction of numbers (\ml{-}), 56, 88, 90, 92
\item {\tt Sum}, 57, 92
\item {\SVal} (special values), 55
\item $\Supp$ (support), 40, 41
\item $\sv$ (special value), 55
\item symbol, 4
\item syntax, 3, 13, 55, 68, 82
\indexspace
\parbox{65mm}{\hfil{\large\bf T}\hfil}
\indexspace
\item $\t$ (type name), 21, 24, 27, 31, 33, 36, 38--40, 42, 50, 89
\item $\T$ (type name set), 23, 38
\item $\TE$ (type environment), 23, 27, 32, 33, 42, 50, 69
\item \THEN, 3, 79, 84
\item $\topdec$ (top-level declaration), 15, 19, 53, 73, 77
\subitem in program, 75, 76
\item TopDec (top-level declarations), 15
\item top-level declaration, 1, 15, 19, 53, 73, 77
\item $\TRUE$, 87--89
\item truncation of reals (\ml{floor}), 56, 88, 91
\item tuple, 79, 80, 84, 86
\item tuple type, 80, 86
\item $\ty$ (type expression), 8, 11, 35, 55, 80, 86
\item Ty (type expressions), 8, 11, 55
\item $\tycon$ (type constructor), 4, 10, 11, 18, 22, 27, 32, 33, 35, 39, 42, 50, 51, 89
\item $\TyCon$ (type constructors), 4
\item $\TyEnv$ (type environments), 23
\item $\TyNames$ (type names), 21
\item $\TyNamesFcn$ (free type names), 22, 45
\item $\TyNameSets$ (type name sets), 23
\item $\typbind$ (type binding), 8, 10, 32, 55, 85
\item TypBind (type bindings), 8, 55
\item $\typdesc$ (type description), 15, 18, 50, 68
\item TypDesc (type descriptions), 15, 68
\item type ($\tau$), 22, 24--26, 28--30, 34, 35
\item $\Type$ (types), 23
\item $\TYPE$, 3, 10, 18, 31, 49, 51, 55, 68, 85
\item $\scontype$ (function on special constants), 22, 28, 34
\item type binding, 8, 10, 32, 55, 85
\item type constraint (\verb+:+) 
\subitem in expression, 10, 29, 55, 84
\subitem in pattern, 11, 35, 55, 86
\item type construction, 11, 35
\item type constructor ($\tycon$), 4, 10, 11, 18, 22, 27, 32, 33, 35, 39, 42, 50, 51, 89
\item type constructor name (see type name) 
\item type declaration, 10, 31, 55, 85
\item type description ($\typdesc$), 15, 18, 50, 68
\item type environment ($\TE$), 23, 27, 32, 33, 42, 50, 69
\item type explication, 41, 42, 46, 48, 53
\item type-explicit signature (see type explication) 
\item type expression, 8, 11, 35, 55, 80, 86
\item type-expression row ($\labtys$), 8, 11, 36, 55, 86
\item type function ($\typefcn$), 23, 24, 27, 32, 39, 40, 42, 50, 51, 89
\item type name ($\t$), 21, 24, 27, 31, 33, 36, 38--40, 42, 50, 89
\item type name set, 23, 38
\item type realisation ($\tyrea$), 40
\item type scheme ($\tych$), 23, 25, 26, 28, 34, 42, 50, 88, 89
\item type specification, 18, 49, 68
\item type structure $(\theta,\CE)$, 23, 27, 31--33, 35, 39, 42, 49--51, 88, 89
\item type variable ($\tyvar$, $\alpha$), 5, 11, 21
\subitem in type expression, 11, 35, 86
\subitem equality, 5, 21, 24, 25
\subitem imperative, 5, 21, 22, 24--26, 31, 33, 36, 53
\subitem applicative, 5, 21, 22, 24--26, 31, 33
\subitem explicit, 25, 30, 31
\item type vector ($\tauk$), 23, 24
\item $\TypeFcn$ (type functions), 23
\item $\TypeScheme$ (type schemes), 23
\item $\labtys$ (type-expression row), 8, 11, 36, 55, 86
\item TyRow (type-expression rows), 8, 11, 55
\item $\TyStr$ (type structures), 23
\item $\tyvar$ (see type variable) 
\item $\TyVar$ (type variables), 4, 21
\item $\TyVarFcn$ (free type variables), 22
\item $\tyvarseq$ (type variable sequence), 8
\item $\TyVarSet$, 23
\indexspace
\parbox{65mm}{\hfil{\large\bf U}\hfil}
\indexspace
\item $\U$ (explicit type variables), 23--25, 31
\item $\UNIT$, 89
\item unguarded type variable, 25
\indexspace
\parbox{65mm}{\hfil{\large\bf V}\hfil}
\indexspace
\item $\V$ (value), 57, 60--63
\item $\sconval$ (function on special constants), 55, 60, 65
\item $\Val$ (values), 57
\item $\VAL$, 3, 10, 18, 31, 49, 64, 72, 85
\item $\valbind$ (value binding), 8, 10, 25, 26, 31, 32, 64, 85
\subitem simple, 10, 32, 64, 85
\subitem recursive, 10, 32, 64, 85
\item Valbind (value bindings), 8
\item $\valdesc$ (value description), 15, 18, 50, 73
\item ValDesc (value descriptions), 15
\item value binding ($\valbind$), 8, 10, 25, 26, 31, 32, 64, 85
\subitem simple, 10, 32, 64, 85
\subitem recursive, 10, 32, 64, 85
\item value constant ($\con$) 
\subitem in pattern, 11, 34, 65, 86
\item value constructor ($\con$), 4
\subitem as atomic expression, 10, 28, 61, 84
\subitem scope, 5, 22
\item value construction 
\subitem in pattern, 11, 35, 66, 86
\subitem infixed, in pattern, 11, 86
\item value declaration, 10, 25, 31, 64, 85
\item value description ($\valdesc$), 15, 18, 50, 73
\item value variable ($\var$), 4
\subitem as atomic expression, 10, 28, 60, 84
\subitem in pattern, 11, 34, 65, 86
\item value specification, 18, 49, 72
\item $\var$ (see value variable) 
\item $\Var$ (value variables), 4
\item $\VarEnv$ (variable environments), 23, 57
\item variable (see value variable) 
\item variable environment ($\VE$) 
\subitem static, 23--22, 26, 31--35, 42, 50, 69, 88, 89
\subitem dynamic, 57, 58, 64--66, 69, 90
\item $\vars$ (set of value variables), 68, 73
\item $\VE$ (see variable environment) 
\item via $\rea$, 42, 54
\item view of a structure, 47, 51, 68, 70, 71
\indexspace
\parbox{65mm}{\hfil{\large\bf W}\hfil}
\indexspace
\item well-formed 
\subitem assembly, 39, 40
\subitem functor signature, 39
\subitem signature, 39
\subitem type structure, 27
\item \WHILE, 3, 79, 84
\item wildcard pattern (\verb+_+), 11, 34, 65, 86
\item wildcard pattern row (\verb+...+), 3, 11, 34, 36, 66, 86
\item \WITH, 3, 10, 85
\item \WITHTYPE, 3, 78, 80, 85
\indexspace
\parbox{65mm}{\hfil{\large\bf Y}\hfil}
\indexspace
\item $\Yield$, 41
\end{theindex}

\end{document}

 HOW TO REVISE THE INDEX 

The index is made using partly LaTex and partly an ML progam;
the latter is found on the file ``index.ml''. 

When LaTex is run on input ``root.tex'' it produces a file ``root.idx''.
Each line in this file is of the form

               \indexentry{idxkey}{pageref}

where idxkey is a key inserted precisely one place in the document 
(in the form of a LaTex command \index{idxkey})
and pageref is the page number of the page LaTex was printing
by the time it encountered the \index{idxkey} command.

A typical idxkey is 45.1 , which will occur in the TeX file 
somewhere near what produces page 45 in the final document. 
In fact, when the keys were first inserted in the TeX file, 
45.1 would be the first key on page 45, but as the document changes, 
one cannot get the pageref from the idxkey simple by taking the 
prefix of the idxkey up to the full stop.

There are many entries in the index that refer to the same
idxkey. Thus the number of idxkeys has been kept relatively
small, typically 2 or 3 pr page. The basic idea, then, is that
there is an ML program (on file ``index.ml'') which
associates entries in the final index with idxkeys by 
a sequence of expressions

              .......
              item ``handle'' [p``45.4'',``57.3''to``59.1'',p``78.2''];
              item ``{\\it happiness}'' [p''38.1''];
              ......
 
The program will first build a conversion table from idxkeys
(such as ``45.4'') to page references by reading thrugh
``root.idx''. Then it will evaluate all the item and subitem
expressions. The item function produces a line in the
latex file (``index.tex'') using the conversion table.
Thus the above lines may produce

               ...........
               \item ``handle'' 46, 57--58, 78
               \item {\it happiness} 38
               ...........

If insertion of more text in the source files result in 
new page splits, then one should manually check that
the item expressions in ``index.ml'' refer to the
right idxkeys. It may be necessary to change some of the
lists in the item expressions and it may be desirable to
insert new \index commands in the source text. However,
if we for simplicity assume that we simply insert new
material corresponding to a new chapter (not affecting
the page splits in other chapters) then one would proceed
as follows:
First add new item expressions in the index.ml file corre-
sponding to new entries in the index (one will have new
\index commands in the new input, of course). Then one
runs latex on ``root.tex'' to produce the ``root.idx'' file. 
Then one runs index.ml (enter ML and type use ``index.ml'').
Then one runs latex on root again, this time giving the 
correct index.





